\section{LaTeX 语法快速演示}

\subsection{文本格式}
这是一段普通文本。我们可以使用 \textbf{加粗},或者 \textit{斜体}(中文通常用楷体代替斜体)。
换行不是按回车,而是空一行。

这是第二段。

\subsection{列表展示}
\begin{enumerate}
    \item 这是有序列表第一项
    \item 这是第二项
    \begin{itemize}
        \item 这是一个嵌套的无序列表
        \item 列表项自动缩进
    \end{itemize}
\end{enumerate}

\subsection{数学公式}
我们在内核调度算法中使用了如下公式:
$$ T_{next} = \alpha T_{now} + (1-\alpha) T_{predicted} $$
其中 $\alpha$ 是一个平滑因子。

\subsection{代码块}
以下它是 NPUcore 的入口函数:

\begin{lstlisting}[language=Rust, basicstyle=\ttfamily\small, frame=single]
#[no_mangle]
pub fn rust_main() -> ! {
    clear_bss();
    println!("Hello, NPUcore!");
    shut_down();
}
\end{lstlisting}

\subsection{插入图片}
我们在这插个图片试试:
\begin{figure}[htbp]  % htbp 代表尽量按 Here, Top, Bottom, Page 的顺序尝试放置
    \centering  % 居中
    % 假设你的图片在 images 文件夹下
    \includegraphics[width=1\textwidth]{images/123456.png}
    \caption{NPUcore 整体架构图} % 图片标题
    \label{fig:scene} % 这里的 label 是为了给正文引用的
\end{figure}

如图 \ref{fig:scene} 所示... % 这样引用,哪怕图片位置变了,编号也会自动更新
