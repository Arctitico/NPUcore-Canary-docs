\section{Vi 编辑器适配}

为了提升 NPUcore 的实用性,我们将轻量级 vi 编辑器移植到了系统中。
这使得用户可以直接在 NPUcore 上进行文本编辑,极大地增强了系统的交互能力。

\subsection{移植背景}

在操作系统开发和测试过程中,文本编辑器是不可或缺的工具。
虽然 NPUcore-BLOSSOM 已经提供了 kilo 编辑器,但 vi 作为 Unix/Linux 世界中最经典的编辑器,
具有以下优势:

\begin{itemize}
    \item 功能更加完善,支持撤销/重做、搜索替换、标记跳转等高级功能
    \item 符合 POSIX 标准,是许多脚本和工具的默认编辑器
    \item 用户基础广泛,学习成本低
\end{itemize}

\subsection{代码来源}

我们选择从 pshell 项目\cite{pshell} 移植 vi 编辑器。
该项目已经将 BusyBox 中的 vi 实现提取为独立可编译的版本,原始代码约 3800 行 C 代码。

BusyBox vi 的原始版权信息:
\begin{itemize}
    \item 作者:Sterling Huxley
    \item 许可证:GPLv2 or later
\end{itemize}

\subsection{适配工作}

pshell 项目的 vi 原本是为 Raspberry Pi Pico 设计的,依赖 Pico SDK 和 littlefs 文件系统。
为了在 NPUcore 上运行,我们进行了以下适配工作:

\subsubsection{移除硬件依赖}

原代码依赖以下 Pico SDK 头文件,需要全部移除:

\begin{lstlisting}[language=C]
// 移除的依赖
#include "hardware/timer.h"
#include "pico/stdio.h"
#include "pico/time.h"
#include "io.h"
\end{lstlisting}

替换为标准 POSIX 头文件:

\begin{lstlisting}[language=C]
// 添加的 POSIX 头文件
#include <unistd.h>
#include <fcntl.h>
#include <sys/stat.h>
#include <sys/ioctl.h>
#include <termios.h>
#include <poll.h>
\end{lstlisting}

\subsubsection{文件系统 API 替换}

原代码使用 littlefs 文件系统 API,需要替换为 POSIX 标准调用:

\begin{table}[H]
\centering
\begin{tabular}{|l|l|}
\hline
\textbf{原 API (littlefs)} & \textbf{新 API (POSIX)} \\
\hline
\texttt{fs\_file\_open()} & \texttt{open()} \\
\texttt{fs\_file\_read()} & \texttt{read()} \\
\texttt{fs\_file\_write()} & \texttt{write()} \\
\texttt{fs\_file\_close()} & \texttt{close()} \\
\texttt{fs\_stat()} & \texttt{stat()} \\
\texttt{lfs\_file\_t} & \texttt{int} (文件描述符) \\
\texttt{struct lfs\_info} & \texttt{struct stat} \\
\hline
\end{tabular}
\caption{文件系统 API 映射表}
\end{table}

\subsubsection{终端处理}

终端是 vi 编辑器的核心交互界面,需要进行以下适配:\\

\textbf{获取终端尺寸}

原代码使用 Pico 特定的 \texttt{get\_screen\_xy()} 函数,
我们使用 \texttt{ioctl} 系统调用替代:

\begin{lstlisting}[language=C]
static void get_screen_xy(uint32_t* cols, uint32_t* row) {
    struct winsize ws;
    if (ioctl(STDOUT_FILENO, TIOCGWINSZ, &ws) == 0 
        && ws.ws_col > 0 && ws.ws_row > 0) {
        *cols = ws.ws_col;
        *row = ws.ws_row;
    } else {
        *cols = 80;  // 默认值
        *row = 24;
    }
}
\end{lstlisting}

\vspace{0.5cm}

\textbf{键盘输入处理} 

原代码使用 \texttt{getchar\_timeout\_us()} 进行带超时的输入读取。
我们使用 \texttt{poll()} + \texttt{read()} 组合实现相同功能:

\begin{lstlisting}[language=C]
static int safe_poll(uint8_t* buffer, int ms) {
    if (ms < 0) {
        // 阻塞读取
        ssize_t n = read(STDIN_FILENO, buffer, 1);
        return (n > 0) ? 1 : 0;
    } else {
        // 带超时的非阻塞读取
        struct pollfd pfd;
        pfd.fd = STDIN_FILENO;
        pfd.events = POLLIN;
        int ret = poll(&pfd, 1, ms);
        if (ret > 0 && (pfd.revents & POLLIN)) {
            ssize_t n = read(STDIN_FILENO, buffer, 1);
            return (n > 0) ? 1 : 0;
        }
        return 0;
    }
}
\end{lstlisting}

\vspace{0.5cm}

\textbf{终端原始模式} 

vi 需要逐字符读取用户输入,必须将终端设置为原始模式(raw mode):

\begin{lstlisting}[language=C]
static void enable_raw_mode(void) {
    struct termios raw;
    
    if (tcgetattr(STDIN_FILENO, &orig_termios) == 0) {
        termios_saved = 1;
        atexit(disable_raw_mode);
        
        raw = orig_termios;
        // 禁用输入处理
        raw.c_iflag &= ~(BRKINT | ICRNL | INPCK | ISTRIP | IXON);
        // 禁用输出处理
        raw.c_oflag &= ~(OPOST);
        // 8位字符
        raw.c_cflag |= (CS8);
        // 禁用回显和规范模式
        raw.c_lflag &= ~(ECHO | ICANON | IEXTEN | ISIG);
        
        tcsetattr(STDIN_FILENO, TCSAFLUSH, &raw);
    }
}
\end{lstlisting}

\subsubsection{其他修改}

\begin{itemize}
    \item 移除 \texttt{UDATA} 节属性宏,使用普通静态变量
    \item 重命名 \texttt{argc}/\texttt{optind} 变量避免与标准库冲突
    \item 实现简化版 \texttt{full\_path()} 函数
    \item 添加 \texttt{main()} 入口函数
\end{itemize}

\subsection{功能验证}

我们通过实际运行测试验证了 vi 编辑器的核心功能。

首先,在命令行启动 vi 并指定文件名 \texttt{test.txt}:

\begin{lstlisting}[language=bash]
    vi-riscv64 test.txt
\end{lstlisting}

如图 \ref{fig:vi-edit} 所示,进入编辑器后可以正常进行文本输入(插入模式)。
随后使用 \texttt{:wq} 命令保存并退出,如图 \ref{fig:vi-save} 所示。
最后,通过 \texttt{cat} 命令查看文件内容,确认数据已正确写入磁盘,如图 \ref{fig:vi-cat} 所示。

\begin{figure}[H] 
    \centering  
    \includegraphics[width=0.9\textwidth]{images/vi编辑文件.png}
    \caption{在 vi 中编辑文件内容} 
    \label{fig:vi-edit} 
\end{figure}

\begin{figure}[H] 
    \centering  
    \includegraphics[width=0.9\textwidth]{images/vi保存并退出.png}
    \caption{使用 :wq 命令保存并退出} 
    \label{fig:vi-save} 
\end{figure}

\begin{figure}[H] 
    \centering  
    \includegraphics[width=0.9\textwidth]{images/vi验证文件结果.png}
    \caption{验证文件内容正确写入} 
    \label{fig:vi-cat} 
\end{figure}

\subsection{其他应用}

除了 vi 编辑器,我们还成功将 2048 游戏移植到了 NPUcore:
\begin{figure}[H] 
    \centering  
    \includegraphics[width=0.7\textwidth]{images/2048.png}
    \caption{2048 游戏运行截图} 
    \label{fig:2048} 
\end{figure}

成功将 tetris 俄罗斯方块游戏移植到了 NPUcore:
\begin{figure}[H] 
    \centering  
    \includegraphics[width=0.85\textwidth]{images/tetris.png}
    \caption{tetris 俄罗斯方块游戏运行截图} 
    \label{fig:tetris} 
\end{figure}