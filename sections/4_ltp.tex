\section{LTP 测试兼容性改进}

本节将介绍 NPUcore-Canary 为提升 LTP(Linux Test Project)测试通过率所做的一系列改进。
通过系统性地分析测试失败原因并逐一修复,我们将 musl-rv 下的 LTP 测试分数从 319 分提升至 649 分。

\subsection{LTP 测试概述}

LTP 是 Linux 内核的标准测试套件,包含数千个测试用例,覆盖系统调用、文件系统、内存管理、
进程管理等各个方面。通过 LTP 测试是验证操作系统 POSIX 兼容性的重要手段。

\subsection{目前最大的问题}

目前跑 LTP 测试到一半的时候会出现类似这样的错误:

\begin{lstlisting}[language=Rust]
[kernel] panicked at 'Heap allocation error, layout = Layout { size: 4096, align: 1 (1 << 0) }', src/mm/heap_allocator.rs:12:5
\end{lstlisting}

经过分析,我们发现这是由于内存泄漏导致的内存耗尽问题。
该问题导致无法一次性测试所有能通过的 ltp 测试样例,只能选择性地跑一部分测例。

目前 NPUcore-Canary 的临时方案是把 KERNEL\_HEAP\_SIZE 从 128MB 提升到 256MB 以规避该问题,但这并非根本解决方案。
未来如果需要进一步提升 LTP 测试分数,仍需定位并修复内存泄漏问题。

\subsection{/proc}

Linux 的 /proc 文件系统提供了访问内核数据结构的接口,许多用户空间程序依赖它获取进程信息。
NPUcore-BLOSSOM 对 /proc 的支持较为有限,导致多个 LTP 测试失败。

\subsubsection{/proc/self/maps}

\texttt{/proc/self/maps} 文件显示当前进程的内存映射信息,格式如下:

\begin{lstlisting}[language=bash, numbers=none]
起始地址-结束地址 权限 偏移 设备号 inode 路径名
# 示例:
00400000-00452000 r-xp 00000000 08:02 173521 /usr/bin/dbus-daemon
\end{lstlisting}

我们实现了 \texttt{ProcMaps} 结构体,通过遍历进程的内存区域动态生成映射信息:

\begin{lstlisting}[language=Rust]
pub struct ProcMaps {
    pub pid: Option<usize>,  // None 表示当前进程
    pub offset: Mutex<usize>,
}

impl ProcMaps {
    fn get_maps_string(&self) -> String {
        let task = if let Some(pid) = self.pid {
            find_task_by_pid(pid)
        } else {
            current_task()
        };
        let vm = task.vm.lock();
        let mut result = String::new();

        for area in vm.get_areas() {
            let start_addr = area.inner.get_start().0 * PAGE_SIZE;
            let end_addr = area.inner.get_end().0 * PAGE_SIZE;
            
            // 构建权限字符串 (r/w/x + s/p)
            let mut perms = String::with_capacity(4);
            let perm = area.map_perm;
            perms.push(if perm.contains(MapPermission::R) { 'r' } else { '-' });
            perms.push(if perm.contains(MapPermission::W) { 'w' } else { '-' });
            perms.push(if perm.contains(MapPermission::X) { 'x' } else { '-' });
            perms.push(if area.is_shared { 's' } else { 'p' });
            
            result.push_str(&format!(
                "{:08x}-{:08x} {} {:08x} {:02x}:{:02x} {} {}\n",
                start_addr, end_addr, perms, offset, 
                dev_major, dev_minor, inode, pathname
            ));
        }
        result
    }
}
\end{lstlisting}

\subsubsection{/proc/[pid]/stat}

\texttt{/proc/[pid]/stat} 文件以单行形式提供进程状态信息,包括 PID、进程名、状态、
父进程 PID、进程组 ID 等字段。这对于 \texttt{getpgid01}、\texttt{wait401} 等测试至关重要。

\begin{lstlisting}[language=Rust]
pub struct ProcPidStat {
    pub pid: usize,
    pub offset: Mutex<usize>,
}

impl ProcPidStat {
    fn get_stat_string(&self) -> String {
        if let Some(task) = find_task_by_pid(self.pid) {
            let inner = task.acquire_inner_lock();
            let pid = task.pid.0;
            let ppid = inner.parent.as_ref()
                .and_then(|p| p.upgrade())
                .map(|p| p.pid.0).unwrap_or(0);
            let pgid = inner.pgid;
            
            // 进程状态映射
            let state = match inner.task_status {
                TaskStatus::Ready | TaskStatus::Running => 'R',
                TaskStatus::Interruptible => 'S',
                TaskStatus::Zombie => 'Z',
            };
            
            // Linux stat 格式: pid (comm) state ppid pgrp session ...
            format!("{} (process) {} {} {} {} 0 0 ...\n",
                pid, state, ppid, pgid, task.tgid)
        } else {
            String::new()
        }
    }
}
\end{lstlisting}

\subsubsection{/proc/[pid]/oom\_score\_adj}

OOM(Out of Memory)killer 使用 \texttt{oom\_score\_adj} 来调整进程被杀死的优先级。
我们在 TCB 中添加了 \texttt{oom\_score\_adj} 字段(范围 -1000 到 1000),
并实现了对应的读写接口。

\begin{lstlisting}[language=Rust]
/// 范围: -1000 (永不被杀) 到 1000 (优先被杀)
pub struct OomScoreAdj {
    pub pid: Option<usize>,
    pub offset: Mutex<usize>,
}

impl OomScoreAdj {
    fn get_oom_score_string(&self) -> String {
        if let Some(task) = self.get_target_task() {
            let inner = task.acquire_inner_lock();
            format!("{}\n", inner.oom_score_adj)
        } else {
            String::from("0\n")
        }
    }

    fn set_oom_score_from_string(&self, input: &str) -> Result<(), ()> {
        let value = input.trim().parse::<i16>().map_err(|_| ())?;
        if value >= -1000 && value <= 1000 {
            if let Some(task) = self.get_target_task() {
                task.acquire_inner_lock().oom_score_adj = value;
                return Ok(());
            }
        }
        Err(())
    }
}

// TCB 中的字段定义
pub struct TaskControlBlockInner {
    // ...
    pub oom_score_adj: i16,  // 默认为 0,fork 时继承父进程的值
}
\end{lstlisting}

\subsection{时间子系统完善}

时间相关功能是许多 LTP 测试的基础,我们进行了以下改进:

\subsubsection{Goldfish RTC 驱动}

QEMU 平台提供了 Goldfish RTC 设备用于获取真实世界时间。
我们实现了该驱动,使内核能够在启动时同步 Unix 时间戳,
而不是从 0 开始计时。

\begin{lstlisting}[language=Rust]
/// 初始化Unix时间(从Goldfish RTC读取)
pub fn init_unix_time() {
    // Goldfish RTC寄存器: 0x00=TIME_LOW, 0x04=TIME_HIGH
    const GOLDFISH_RTC_BASE: usize = 0x10_1000;
    
    unsafe {
        let time_low = read_volatile(GOLDFISH_RTC_BASE as *const u32) as u64;
        let time_high = read_volatile((GOLDFISH_RTC_BASE + 4) as *const u32) as u64;
        let rtc_ns = (time_high << 32) | time_low;
        let rtc_sec = (rtc_ns / 1_000_000_000) as usize;
        
        let uptime_sec = get_time() / get_clock_freq();
        let boot_unix_time = rtc_sec.saturating_sub(uptime_sec);
        UNIX_TIME_OFFSET.store(boot_unix_time, AtomicOrdering::Relaxed);
    }
}
\end{lstlisting}

\subsubsection{64位时间戳支持}

ext4 文件系统的 inode 结构中,时间戳字段(atime、mtime、ctime)使用 32 位存储秒数,
额外的 \texttt{i\_*\_extra} 字段存储纳秒和扩展纪元。我们完善了这部分的解析逻辑,
支持完整的 64 位时间戳,避免 2038 年问题。

\subsubsection{clock\_nanosleep 改进}

修复了 \texttt{clock\_nanosleep} 系统调用的参数验证:
\begin{itemize}
    \item 验证 \texttt{tv\_nsec} 范围(0 到 999999999)
    \item 对不支持的时钟类型返回 \texttt{ENOTSUP}
\end{itemize}

\subsection{文件系统改进}

\subsubsection{ext4 ftruncate 零填充}

POSIX 标准要求:当使用 \texttt{ftruncate} 扩展文件大小时,扩展部分必须填充为零。
原实现仅更新 inode 大小,未对扩展区域进行零填充。
我们添加了 \texttt{clear\_at\_block\_cache()} 方法实现这一功能:

\begin{lstlisting}[language=Rust]
// truncate 时,若文件扩展则清零扩展区域
if new_size > old_size {
    self.clear_at_block_cache(old_size, new_size - old_size);
}

fn clear_at_block_cache(&self, offset: usize, length: usize) -> usize {
    let mut start = offset;
    let end = offset + length;
    loop {
        let end_current_block = ((start / CACHE_SZ + 1) * CACHE_SZ).min(end);
        let block_write_size = end_current_block - start;
        // 将块缓存中对应区域填充为 0
        self.file_cache_manager.get_cache(...).lock()
            .modify(0, |data: &mut [u8; PAGE_SIZE]| {
                data[start % CACHE_SZ..][..block_write_size].fill(0);
            });
        if end_current_block == end { break; }
        start = end_current_block;
    }
}
\end{lstlisting}

\subsubsection{chown 实现}

实现了完整的 \texttt{fchownat} 系统调用:

\begin{lstlisting}[language=Rust]
pub fn sys_fchownat(dirfd: i32, pathname: *const u8, 
                    owner: u32, group: u32, _flags: i32) -> isize {
    // ... 路径解析 ...
    let stat = target_file.file.get_stat();
    // -1 (0xFFFFFFFF) 表示保持原值不变
    let new_uid = if owner == u32::MAX { stat.get_uid() } else { owner };
    let new_gid = if group == u32::MAX { stat.get_gid() } else { group };
    target_file.file.set_owner(new_uid, new_gid)
}
\end{lstlisting}

\subsubsection{symlinkat 系统调用}

实现了 \texttt{symlinkat} 系统调用(syscall 36),支持创建符号链接。

\subsubsection{路径验证增强}

添加了完整的路径验证逻辑:
\begin{itemize}
    \item \texttt{PATH\_MAX}(4096):路径总长度限制
    \item \texttt{NAME\_MAX}(255):单个路径组件长度限制
    \item 符号链接循环检测,超过限制返回 \texttt{ELOOP}
\end{itemize}

\subsection{内存管理增强}

\subsubsection{MAP\_SHARED 支持}

实现了 \texttt{mmap} 的 \texttt{MAP\_SHARED} 标志支持。在 \texttt{MapArea} 中添加标志位,
fork 时对共享映射区域共享物理页面而非写时复制:

\begin{lstlisting}[language=Rust]
pub struct MapArea {
    pub is_shared: bool,     // MAP_SHARED 标志
    pub wipe_on_fork: bool,  // MADV_WIPEONFORK 标志
    // ...
}

// fork 时的处理逻辑
if src_area.is_shared {
    // 共享映射:直接共享物理页面
    new_area.map_shared(src_area);
} else if src_area.wipe_on_fork {
    // WIPEONFORK:子进程获得零填充页面
    new_area.map_zeroed();
} else {
    // 私有映射:写时复制 (COW)
    new_area.map_cow(src_area);
}
\end{lstlisting}

\subsubsection{madvise 改进}

实现了 \texttt{MADV\_WIPEONFORK} 和 \texttt{MADV\_KEEPONFORK} 标志:

\begin{lstlisting}[language=Rust]
pub fn madvise(&mut self, addr: usize, length: usize, advice: u32) -> Result<(), isize> {
    const MADV_WIPEONFORK: u32 = 18;
    const MADV_KEEPONFORK: u32 = 19;
    
    match advice {
        MADV_WIPEONFORK | MADV_KEEPONFORK => {
            let wipe = advice == MADV_WIPEONFORK;
            for area in self.areas.iter_mut() {
                // 仅对私有匿名页生效,跳过共享/文件映射
                if area.is_shared || area.map_file.is_some() { continue; }
                area.wipe_on_fork = wipe;
            }
            Ok(())
        }
        // ...
    }
}
\end{lstlisting}

\subsubsection{brk 系统调用修复}

修复了 \texttt{brk} 系统调用的若干问题:
\begin{itemize}
    \item 正确处理非页对齐的堆增长
    \item 添加溢出保护
\end{itemize}

\subsection{系统调用修复}

\subsubsection{dup2/dup3 修复}

\begin{itemize}
    \item 修正 \texttt{SYSCALL\_DUP3} 的系统调用号(从 20 改为 24)
    \item 修复 \texttt{newfd} 无效时应返回 \texttt{EBADF} 而非 \texttt{EMFILE}
    \item 在新架构上,glibc/musl 将 \texttt{dup2} 转换为 \texttt{dup3} 调用
\end{itemize}

\subsubsection{statx 改进}

\begin{itemize}
    \item 添加 \texttt{STATX\_\_RESERVED} 掩码验证
    \item 支持 \texttt{AT\_EMPTY\_PATH} 标志
    \item 支持 \texttt{AT\_SYMLINK\_NOFOLLOW} 标志
\end{itemize}

\subsubsection{fcntl 文件锁}

实现了 \texttt{F\_GETLK}、\texttt{F\_SETLK}、\texttt{F\_SETLKW} 命令,
定义了与 Linux ABI 兼容的 \texttt{Flock} 结构:

\begin{lstlisting}[language=Rust]
pub const F_RDLCK: i16 = 0;  // 读锁(共享)
pub const F_WRLCK: i16 = 1;  // 写锁(独占)
pub const F_UNLCK: i16 = 2;  // 解锁

#[repr(C)]
pub struct Flock {
    pub l_type: i16,    // 锁类型
    pub l_whence: i16,  // 起始位置基准
    pub l_start: i64,   // 锁起始偏移
    pub l_len: i64,     // 锁定字节数,0 表示到文件末尾
    pub l_pid: i32,     // 持有锁的进程 PID(F_GETLK 返回)
}
\end{lstlisting}

\subsubsection{新增系统调用}

\begin{table}[H]
\centering
\begin{tabular}{|c|l|l|}
\hline
\textbf{系统调用号} & \textbf{名称} & \textbf{功能} \\
\hline
33 & mknodat & 创建设备文件或 FIFO \\
\hline
36 & symlinkat & 创建符号链接 \\
\hline
44 & fstatfs & 获取文件系统统计信息 \\
\hline
47 & fallocate & 预分配文件空间 \\
\hline
81 & sync & 同步文件系统缓存 \\
\hline
92 & personality & 设置进程执行域 \\
\hline
114 & clock\_getres & 获取时钟分辨率 \\
\hline
144 & setgid & 设置进程组 ID \\
\hline
146 & setuid & 设置进程用户 ID \\
\hline
\end{tabular}
\caption{新增系统调用}
\end{table}

\subsection{错误处理规范化}

为了符合 POSIX 标准,我们对错误处理进行了全面的规范化:

\subsubsection{标准错误码}

\begin{itemize}
    \item \texttt{pread/pwrite}:负偏移返回 \texttt{EINVAL}
    \item \texttt{splice}:文件到文件的 splice 返回 \texttt{EINVAL}
    \item \texttt{getcwd}:按 POSIX 标准调整错误检查顺序
\end{itemize}

\subsubsection{边界检查与溢出保护}

\begin{itemize}
    \item \texttt{translated\_byte\_buffer}:添加地址溢出保护
    \item \texttt{contains\_valid\_buffer}:对溢出地址返回 false
\end{itemize}
