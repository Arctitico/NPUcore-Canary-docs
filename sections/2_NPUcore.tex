\section{NPUcore 内核架构分析}

本节将对 NPUcore-Canary 内核项目的基准版本 NPUcore-BLOSSOM 进行全面分析与介绍。
NPUcore-BLOSSOM 是西北工业大学 2025 年参加全国大学生计算机系统能力大赛——操作系统设计赛的参赛作品,
在吸取了 NPUcore+\cite{npucore-plus} (2023年二等奖) 和 NPUcore-IMPACT!!!\cite{npucore-impact} (2024年一等奖) 两个优秀的 NPUcore 家族作品的经验后,
具备了完善的 EXT4 文件系统支持、HAL 层双架构支持(LoongArch64 与 RISC-V)以及丰富的应用程序兼容性。

本节中的插图均来自 NPUcore-BLOSSOM 项目仓库中的设计文档。

\subsection{整体架构概述}

NPUcore 采用分层架构设计,自底向上分为平台层、硬件抽象层(HAL)、内核层和应用层四个主要层次。
这种分层设计使得内核具有良好的可移植性和可维护性。

\begin{figure}[htbp]  
    \centering  
    \includegraphics[width=1\textwidth]{images/BLOSSOM/整体架构图.png}
    \caption{NPUcore 整体架构图} 
    \label{fig:NPUcore architecture} 
\end{figure}

\textbf{平台层}是系统的基础支撑,为上层提供硬件资源和运行环境。NPUcore 支持多种平台:
\begin{itemize}
    \item \textbf{虚拟平台}:\\QEMU 模拟器(同时支持 LoongArch64 和 RISC-V)
    \item \textbf{物理平台}:\\K210 Board、fu740 Board、VisionFive2 Board(RISC-V);\\2K1000 Board(LoongArch64)
\end{itemize}

\textbf{硬件抽象层(HAL)}是 NPUcore-BLOSSOM 团队按照2025全国赛要求新增的关键中间层,通过这层抽象,NPUcore 从单一指令集内核转变为同时支持 LoongArch64 和 RISC-V 两种指令集的内核,大大提高了系统的扩展性。
HAL 层的核心由 arch 模块、config 模块和 platform 模块组成:
\begin{itemize}
    \item \textbf{arch 模块}:实现不同处理器架构的底层操作,包括寄存器操作、trap 处理、内存管理、上下文切换等
    \item \textbf{config 模块}:为特定架构在 QEMU 虚拟环境下的运行提供参数配置
    \item \textbf{platform 模块}:适配不同硬件平台,提供底层操作接口以屏蔽硬件差异
\end{itemize}

\textbf{内核层}是系统的核心,负责管理进程、内存、文件系统等关键资源,并提供进程间通信、中断处理等核心功能。

\textbf{应用层}是系统与用户交互的接口,NPUcore 在实现完整系统调用的基础上,能够通过多项测试程序并支持额外的轻量级应用(如 kilo 文本编辑器、终端版俄罗斯方块等)。

\subsection{进程管理模块}

进程是在系统中运行的程序实例,其生命周期包括创建、就绪、阻塞、运行和退出五个阶段。
NPUcore 采用如图 \ref{fig:NPUcore process management} 所示的进程管理流程。

\begin{figure}[htbp] 
    \centering  
    \includegraphics[width=0.6\textwidth]{images/BLOSSOM/进程管理调度图.png}
    \caption{NPUcore 进程管理流程图} 
    \label{fig:NPUcore process management} 
\end{figure}

\subsubsection{进程的创建}

NPUcore 中进程的创建分为三个步骤:
\begin{enumerate}
    \item \textbf{初始进程创建}:内核初始化完毕后,从系统文件中找到 initproc 的 ELF 文件,调用 TaskControlBlock 的 new() 方法创建第一个初始进程
    \item \textbf{clone 系统调用}:其余所有进程均由初始进程通过 clone 系统调用克隆而来,initproc 是所有进程的祖先
    \item \textbf{execve 系统调用}:新进程通过 execve 系统调用加载独立的程序代码和数据,完成进程的"变身"
\end{enumerate}

进程控制块(TaskControlBlock)包含进程的所有状态信息,包括 PID、TID、内核栈、用户栈、文件描述符表、内存映射、信号处理等。
采用 RAII(资源获取即初始化)思想管理资源,当进程退出时自动释放所有相关资源。

\subsubsection{进程的调度}

NPUcore 实现了阻塞式进程调度,核心数据结构包括:
\begin{itemize}
    \item \textbf{TaskManager}:任务管理器,包含就绪队列(ready\_queue)和可中断睡眠队列(interruptible\_queue)
    \item \textbf{WaitQueue}:等待队列,存放等待特定事件的进程弱引用
    \item \textbf{TimeoutWaitQueue}:超时等待池,使用二叉堆实现,支持定时唤醒
\end{itemize}

调度器采用轮询机制,在操作系统运行的任一时刻都尝试从 idle 流程切换到下一个进程。
进程可通过以下方式被唤醒:
\begin{itemize}
    \item 操作系统主动唤醒处于 interruptible 状态的进程
    \item 定时器中断触发,自动唤醒超时等待池中的进程
\end{itemize}

\subsection{内存管理模块}

内存管理是操作系统的核心功能之一,NPUcore 实现了完整的虚拟内存机制,
包括地址空间抽象、多级页表、物理内存分配等功能。

\subsubsection{内核虚拟地址空间}

地址空间(Address Space)通过在内核中建立虚拟地址到物理地址的映射机制,
为应用程序构建安全、统一的虚拟内存环境。NPUcore 采用图 \ref{fig:NPUcore virtual address space} 所示的内核虚拟地址空间分配策略。

\begin{figure}[htbp] 
    \centering  
    \includegraphics[width=0.5\textwidth]{images/BLOSSOM/内核虚拟地址空间图.png}
    \caption{NPUcore 内核虚拟地址空间图} 
    \label{fig:NPUcore virtual address space} 
\end{figure}

各区域的功能如下:
\begin{itemize}
    \item \textbf{trampoline}:跳板页面,位于地址空间最高虚拟页面,用于用户态和内核态之间的安全切换
    \item \textbf{User Stack}:用户栈区域,由高地址向低地址增长
    \item \textbf{Guard Page}:保护页面,位于相邻用户栈之间,用于检测栈溢出
    \item \textbf{kernel program}:内核程序加载区,采用恒等映射
    \item \textbf{temporary storage area}:临时存储区,用于 exec 系统调用时的文件加载
\end{itemize}

\subsubsection{物理地址空间分布}

在 RISC-V 架构下,QEMU 平台的地址空间分布如图所示:

\begin{figure}[H] 
    \centering  
    \includegraphics[width=0.5\textwidth]{images/BLOSSOM/RISCV架构下QEMU平台地址分布图.png}
    \caption{RISC-V 架构下 QEMU 平台地址分布图} 
    \label{fig:RISC-V QEMU address distribution} 
\end{figure}

内核采用恒等映射方式映射自身的代码段(.text)、只读数据段(.rodata)、数据段(.data)和 BSS 段(.bss),
确保在启用页表机制后仍能直接访问自身各个段。

\subsubsection{地址空间数据结构}

NPUcore 中地址空间的核心数据结构如图所示:

\begin{figure}[H] 
    \centering  
    \includegraphics[width=0.5\textwidth]{images/BLOSSOM/地址空间数据结构图.png}
    \caption{NPUcore 地址空间数据结构图} 
    \label{fig:NPUcore address space data structure} 
\end{figure}

地址空间由页表和一系列内存映射区域构成:
\begin{itemize}
    \item \textbf{MemorySet}:地址空间的顶层抽象,包含页表和 MapArea 向量
    \item \textbf{MapArea}:逻辑段,表示一段连续的虚拟内存区域
    \item \textbf{LinearMap}:线性映射,包含虚拟页号范围和物理页帧追踪器
    \item \textbf{FrameTracker}:物理页帧追踪器,实现 RAII 自动资源管理
\end{itemize}

\subsubsection{多级页表机制}

在 RISC-V 架构下,NPUcore 实现了 SV39 三级页表机制。SV39 支持 39 位虚拟内存空间,
每页 4KB,虚拟地址的高 27 位划分为三级页号,每级 512 个页表项。地址转换过程如图所示:

\begin{figure}[H] 
    \centering  
    \includegraphics[width=1\textwidth]{images/BLOSSOM/RISCVsv39三级索引地址转换示意图.png}
    \caption{RISC-V sv39 三级索引地址转换示意图} 
    \label{fig:RISC-V sv39 address translation} 
\end{figure}

页表项(PTE)包含物理页号和标志位,标志位包括:
\begin{itemize}
    \item V(Valid):有效位,仅当为 1 时页表项合法
    \item R/W/X(Read/Write/Execute):控制页面的读/写/执行权限
    \item U(User):控制页面是否允许用户态访问
    \item G(Global):全局标志位,任务切换时 TLB 不失效
    \item A(Accessed):访问位,记录页面是否被访问过
    \item D(Dirty):脏位,记录页面是否被修改过
\end{itemize}

在 LoongArch64 架构下,NPUcore 实现了 LAMex 分页机制,同样支持 52 位虚拟内存空间和多级页表索引。

\subsubsection{页面错误处理}

NPUcore 实现了完整的页面错误(Page Fault)处理机制,支持懒分配和写时复制(Copy-on-Write):

\begin{figure}[H] 
    \centering  
    \includegraphics[width=0.9\textwidth]{images/BLOSSOM/页面错误处理流程图.png}
    \caption{NPUcore 页面错误处理流程图} 
    \label{fig:NPUcore page fault handling} 
\end{figure}

页面错误的处理逻辑:
\begin{enumerate}
    \item 检查发生错误的虚拟地址是否在合法的地址空间范围内
    \item 若页面未映射:执行懒分配,分配物理页帧并建立映射
    \item 若页面已映射但触发写错误:执行写时复制,复制页面内容到新的物理页帧
    \item 若地址非法或权限不足:发送 SIGSEGV 或 SIGBUS 信号
\end{enumerate}

\subsubsection{物理内存分配}

NPUcore 采用栈式物理内存分配器(StackFrameAllocator),具有以下特点:
\begin{itemize}
    \item 使用全局物理内存分配器确保内存分配的一致性和线程安全
    \item 采用"后进先出"策略管理空闲物理页帧
    \item 通过 FrameTracker 实现 RAII,页帧在追踪器析构时自动回收
    \item 支持不初始化分配(alloc\_uninit)以提高性能
\end{itemize}

\subsection{文件系统模块}

NPUcore 实现了完整的虚拟文件系统(VFS)层,支持 FAT32 和 EXT4 两种文件系统格式。

\subsubsection{文件系统架构}

\begin{figure}[H] 
    \centering  
    \includegraphics[width=0.6\textwidth]{images/BLOSSOM/文件系统架构图.png}
    \caption{NPUcore 文件系统架构图} 
    \label{fig:NPUcore file system architecture} 
\end{figure}

文件系统采用分层设计:
\begin{itemize}
    \item \textbf{VFS 层}:提供统一的文件操作接口,屏蔽底层文件系统差异
    \item \textbf{具体文件系统层}:实现 FAT32 和 EXT4 的具体操作
    \item \textbf{块缓存层}:提供磁盘块的缓存管理
    \item \textbf{块设备层}:与底层存储设备交互
\end{itemize}

\subsubsection{VFS 模块组成}

VFS 层包含以下核心模块:

\begin{figure}[H] 
    \centering  
    \includegraphics[width=0.7\textwidth]{images/BLOSSOM/文件系统模块图.png}
    \caption{NPUcore 文件系统模块图} 
    \label{fig:NPUcore file system modules} 
\end{figure}

\begin{itemize}
    \item \textbf{超级块(super block)}:保存文件系统的所有元数据
    \item \textbf{目录项模块}:管理路径的目录项,采用树形结构组织
    \item \textbf{inode 模块}:管理具体文件,是文件的唯一标识
    \item \textbf{打开文件列表模块}:维护所有已打开的文件句柄
    \item \textbf{file\_operations 模块}:包含所有可用的文件操作函数指针
    \item \textbf{address\_space 模块}:管理文件在页缓存中的物理页
\end{itemize}

\subsubsection{文件抽象}

NPUcore 定义了统一的 File trait,提供以下核心操作:
\begin{itemize}
    \item 基本操作:read、write、readable、writable
    \item 元数据操作:get\_size、get\_stat、get\_file\_type
    \item 目录操作:open、create、unlink、get\_dirent
    \item 偏移管理:lseek、get\_offset
    \item 大小管理:modify\_size、truncate\_size
\end{itemize}

支持的文件类型包括:普通文件、目录、管道(FIFO)、字符设备、块设备、套接字和符号链接。

\subsection{系统调用模块}

系统调用是应用程序与操作系统内核交互的标准接口。NPUcore-BLOSSOM 实现了 POSIX 标准的系统调用接口,
支持 90 余个系统调用。

\begin{figure}[H] 
    \centering  
    \includegraphics[width=0.7\textwidth]{images/BLOSSOM/系统调用整体架构图.png}
    \caption{NPUcore 系统调用整体架构图} 
    \label{fig:NPUcore system call architecture} 
\end{figure}

\subsubsection{系统调用机制}

在 RISC-V 架构中,系统调用通过 ecall 指令触发。当用户态程序执行 ecall 指令时,
CPU 跳转到由 STVEC 寄存器指定的中断处理程序入口地址。系统调用的具体流程:
\begin{enumerate}
    \item 用户态程序调用库函数(如 write)
    \item 库函数调用 syscall() 函数,将参数放入寄存器
    \item 执行 ecall 指令,触发 trap 进入内核态
    \item 内核的 trap\_handler 根据系统调用号分发处理
    \item 执行完成后通过 trap\_return 返回用户态
\end{enumerate}

在 LoongArch 架构中,系统调用通过 syscall 指令触发,CPU 自动切换到内核态并跳转到 CSR\_EBASE 寄存器指定的处理程序入口。

\subsubsection{核心系统调用}

NPUcore-BLOSSOM 已实现的核心系统调用包括:

\textbf{进程管理类}:
\begin{itemize}
    \item fork/clone:创建子进程,支持多种克隆标志
    \item execve:加载并执行新程序,支持 ELF 可执行文件和脚本文件
    \item wait4/waitpid:等待子进程状态变化
    \item exit/exit\_group:进程退出
\end{itemize}

\textbf{内存管理类}:
\begin{itemize}
    \item brk/sbrk:调整堆空间大小
    \item mmap/munmap:内存映射和解除映射
    \item mprotect:修改内存区域保护属性
\end{itemize}

\textbf{文件操作类}:
\begin{itemize}
    \item open/close:打开和关闭文件
    \item read/write:读写文件
    \item lseek:移动文件读写位置
    \item dup/dup2/dup3:复制文件描述符
    \item pipe/pipe2:创建管道
\end{itemize}

\textbf{信号处理类}:
\begin{itemize}
    \item sigaction:设置信号处理程序
    \item sigprocmask:设置信号屏蔽字
    \item kill:发送信号
\end{itemize}

\textbf{NPUcore-Canary 项目在 NPUcore-BLOSSOM 的基础上进行进一步的优化和扩展,
后续章节将详细介绍我们所做的改进工作。}